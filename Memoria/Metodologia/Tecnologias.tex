\section{Tecnologías a utilizar}
\subsection{Diseño de la arquitectura}
La arquitectura consta de una red privada\footnote{La red será cableada mediante cables UTP de categoría 5e.}, interconectada mediante un switch, las tarjetas Zybo y un ordenador que será denominado "monitor" ya que será el encargado de enviar los ficheros iniciales y recibir los ficheros con los datos finales.

Los datos manejados por las tarjetas Zybo estarán cifrados usando el cifrado AES. Para ello se ha empleado un IP\footnote{Intelectual Property.} diseñado y verificado en otro Trabajo de Fin de Grado realizado por Cristian Ambrosio Costoya.

\subsection{Diseño de componentes}
\subsubsection{Ordenador monitor}
El ordenador usado como monitor central es un Toshiba Satellite L750 que cuenta con los siguientes componentes:
\begin{itemize}
	\item Procesador Intel Core I5-3240 quad-core.
	\item 6 Gb de memoria RAM.
	\item Disco duro SSD de 240 Gb.
	\item SO Debian 9 Stretch.
\end{itemize}

\subsubsection{Tarjetas Zybo Zynq 7010}
Los componentes principales de este proyecto son las tarjetas Zybo Zynq 7010 que tienen las siguientes características:
\begin{itemize}
	\item Procesador Cortex-A9 doble núcleo a 650 MHz.
	\item Memoria DDR3 con 8 canales de DMA\footnote{Acceso Directo a Memoria: Permite a cierto tipo de componentes de una computadora acceder a la memoria del sistema para leer o escribir independientemente de la unidad central de procesamiento (CPU) principal.}.
	\item Controladores de periféricos de gran ancho de banda: 1Gb Ethernet, USB 2.0.
	\item Controladores de periféricos de bajo ancho de banda: SPI, UART, CAN, I$^2$C.
	\item Ranura MicroSD (compatible con el sistema de archivos Linux).
	\item Lógica reprogramable equivalente a Artix-7 FPGA:
	\begin{itemize}
		\item 4.400 segmentos lógicos, cada uno con cuatro LUT de 6 entradas y 8 flip-flops.
		\item 512 MB x32 DDR3 con ancho de banda de 1050 Mbps.
		\item Dos cristales de administración de reloj, cada uno con un bucle de fase bloqueada (PLL) y un administrador de reloj de modo mixto (MMCM).
		\item Procesadores digitales de señales de 80 componentes.
		\item Velocidades de reloj interno que exceden los 450MHz.
		\item Convertidor analógico a digital en chip (XADC).
	\end{itemize}
\end{itemize}

Cada tarjeta Zybo incluye una instalación de un sistema operativo Linux para facilitar la gestión de los ficheros y las comunicaciones entre los nodos. El sistema operativo incluido en cada tarjeta SD del proyecto ha sido diseñado en el Trabajo de Fin de Grado de Gabriel Fernando Sánchez Reina, el cual ya incluye el IP cifrador/descifrador del Trabajo de Fin de Grado de Cristian Ambrosio Costoya.

Para la instalación del sistema operativo podemos ver el apéndice \hyperlink{InstalacionLinux}{Instalación de Linux en SD para tarjetas Zybo.} \textcolor{red}{Ahora no se si dejar esta frase o no, porque yo lo que estoy haciendo es clonar la tarjeta de Gabri en otras para que tengan la misma versión de todo y que sean idénticas. Otra posibilidad sería poner el proceso de clonación de las SD en el anexo de la instalación que tengo yo, porque a ese sistema operativo aún tengo que hacerle yo una minimísima modificación para que me deje funcionar a mí. El único problema es que tarda unas dos horas en clonarlo, pero no hay problema con eso.}

\subsubsection{Switch}
El switch utilizado es un Tp-Link TL-SG1024D con las siguientes especificaciones:
\begin{itemize}
	\item Estándares y protocolos: IEEE 802.3i, IEEE 802.3u, IEEE 802.3ab , IEEE 802.3x.
	\item Interfaz: 24 puertos RJ45 a 10/100/1000 Mbps con negociación automática
	(MDI/MDIX automático).
	\item Medios de red: 10BASE-T: cable UTP categoría 3, 4, 5 (100 metros máximo)
	100BASE-TX/1000BASE-T: cable UTP categoría 5, 5e o above cable (máximo 100 metros).
	\item Capacidad de conmutación: 48 Gbps.
	\item Tasa de reenvío de paquetes: 35.7 Mpps.
	\item Tabla de direcciones MAC: 8K.
	\item Certificaciones: FCC, CE, RoHS.
\end{itemize}
