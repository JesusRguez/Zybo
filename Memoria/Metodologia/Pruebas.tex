\section{Pruebas del sistema}
%Aquí describimos los scripts para hacer pruebas e incluir las pruebas que hice para ver su funcionamiento.
\subsection{Hardware}
%\textcolor{red}{Prueba de la infraestructura: Lanzar el script Inicio.sh para que se vea que todas las tarjetas dan ping.}
Para realizar la prueba de la arquitectura de red, lanzaremos el script \hyperlink{ScriptConexion}{\texttt{Inicio.sh}}. Este script, nos dirá qué tarjeta está conectada o desconectada de la red.

\begin{figure}[h]
	\centering
	\includegraphics[scale=0.5]{Metodologia/Pruebas/Prueba_Inicio_sh.png}
	\caption{Prueba de \texttt{Inicio.sh}}
	\label{Prueba de Inicio.sh}
\end{figure}

\subsection{Software}
%\textcolor{red}{Prueba de funcionamiento de la recolección de datos: Una prueba de todo funcionando.}
Para realizar las pruebas de funcionamiento, directamente se usaron los scripts descritos en el \hyperlink{Scripts}{Apéndice B} una vez instalados y configuradas las tarjetas Zybo.

Para ello, se siguió la cadena de conexión desde que el fichero se crea en el ordenador monitor, hasta que vuelve a él después de haber pasado por el resto de nodos de la red.
\begin{enumerate}
	\item Creación del fichero: Creamos un fichero en el ordenador central. Para esta prueba, he creado un fichero en la carpeta ``Descargas'' del ordenador central.

	\item Envío del fichero inicial a la primera tarjeta: Esto lo haremos gracias a la herramienta \texttt{sshpass} de Linux con la siguiente orden:
	\begin{center}
		\texttt{sshpass -p zyboX scp -o StrictHostKeyChecking=no archivoLocal zyboX@zyboX:/home/zyboX/ficheros/recibir}
	\end{center}
	Este proceso lo podemos ver más detallada mente en el \hyperlink{EnvioRecepcionFicheros}{Apéndce A.4}.
	\begin{figure}[h]
		\centering
		\includegraphics[scale=0.5]{Metodologia/Pruebas/Fichero_inicial_en_PC.png}
		\caption{Fichero inicial en el ordenador central y envío al primer nodo}
		\label{Fichero inicial en el ordenador central y envío al primer nodo}
	\end{figure}
\newpage
	\item Recepción del fichero en las tarjetas: Será el script \hyperlink{ScriptRecibiendo}{\texttt{Recibiendo.sh}} el encargado de comprobar la llegada del fichero y actuar en consecuencia cambiándolo de directorio.

	\item Modificación del fichero: El script \hyperlink{ScriptCristian}{\texttt{Cristian.sh}} será el encargado de abrir el fichero, modificarlo en cada una de las tarjetas y dejarlo preparado para su envío al siguiente nodo de la red.
	\begin{figure}[h]
		\centering
		\includegraphics[scale=0.5]{Metodologia/Pruebas/Fichero_en_Zybo1.png}
		\caption{Estado del fichero después de su modificación en Zybo1}
		\label{Estado del fichero después de su modificación en Zybo1}
	\end{figure}
\newpage
	\item Envío del fichero hacia el siguiente nodo: Podemos distinguir dos tipos de envío del mismo fichero. Ambos llevados a cabo por el script \hyperlink{ScriptEnviando}{Enviando.sh}:
	\begin{enumerate}
		\item Tarjeta-Tarjeta: Esta opción se dará cuando la tarjeta actual detecte que la siguiente tarjeta está conectada.
		\item Tarjeta-Ordenador: Esta opción se dará cuando la tarjeta no detecte a la siguiente tarjeta. Entonces, enviará el fichero de datos, de vuelta al ordenador central.
	\end{enumerate}

\newpage
	\item Recepción del fichero en el ordenador central: El fichero final de datos, será recibido en el directorio\\ \texttt{/home/jesus/Vídeos} del ordenador central y contendrá los datos añadidos por todos los nodos de la red.
	\begin{figure}[h]
		\centering
		\includegraphics[scale=0.5]{Metodologia/Pruebas/Fichero_final_en_PC.png}
		\caption{Fichero final en el ordenador central}
		\label{Fichero final en el ordenador central}
	\end{figure}

\end{enumerate}