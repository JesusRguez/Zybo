\section{Pruebas del sistema}
%Aquí describimos los scripts para hacer pruebas e incluir las pruebas que hice para ver su funcionamiento.
\subsection{Hardware}
%\textcolor{red}{Prueba de la infraestructura: Lanzar el script Inicio.sh para que se vea que todas las tarjetas dan ping.}
Para realizar la prueba de la arquitectura de red, lanzaremos el script \hyperlink{ScriptConexion}{\texttt{Inicio.sh}}. Este script, nos dirá qué tarjeta está conectada o desconectada de la red.

\textcolor{red}{Poner la foto del Inicio.sh funcionando.}

\subsection{Software}
%\textcolor{red}{Prueba de funcionamiento de la recolección de datos: Una prueba de todo funcionando.}
Para realizar las pruebas de funcionamiento, directamente se usaron los scripts descritos en el \hyperlink{Scripts}{Apéndice B} una vez instalados y configuradas las tarjetas Zybo.

\textcolor{red}{Poner foto del envío del fichero, su estancia en cada tarjeta Zybo y la vuelta al ordenador habiendo sido modificado por las tarjetas.}

Para ello, se siguió la cadena de conexión desde que el fichero se crea en el ordenador monitor, hasta que vuelve a él después de haber pasado por el resto de nodos de la red. \textcolor{red}{Poner en cada paso, una imagen que ilustre el proceso.}
\begin{enumerate}
	\item Creación del fichero:
	\item Envío del fichero:
	\item Recepción del fichero:
	\item Modificación del fichero:
	\item Envío del fichero hacia el siguiente nodo:
\end{enumerate}