\section{Análisis del sistema}
Una vez que los dispositivos están conectados y se ha creado la infraestructura de red (ver anexo \hyperlink{CreacionInfraestructura}{Creación de una infraestructura de red con tarjetas Zybo}), el sistema es capaz de cifrar los datos, enviarlos por el conjunto de placas que forman la red, descifrar los datos y volverlos a enviar. De esta forma, se consigue que la recolecta de datos sea totalmente cifrada, ya que las comunicaciones están cifradas gracias al protocolo AES y al protocolo SSH mediante el cual se envían y reciben los datos.

\subsection{Hardware}
%Requisitos de la red en si, explicando lo que tendrá que haber y como deberá conectarse, sin decir todavía cómo.
La red estará formada por las tarjetas Zybo, el ordenador central (monitor) y el switch. Cada tarjeta debe tener instalado un sistema operativo Linux para permitir la gestión de ficheros y el uso del protocolo SSH, que será usado en el envío y recepción de ficheros de datos.

\subsection{Software}
%Hablará de los test que habrá que diseñar y luego de los scripts que deberán automatizar el funcionamiento del sistema.
Una vez que se ha montado la red completa, debemos probar que las conexiones entre las tarjetas y el ordenador monitor están habilitadas. Para ello, se ha diseñado un pequeño script con un test de prueba de conectividad (ver anexo \hyperlink{TestConexion}{Test de interconexión de red Zybo}), el cual nos dirá qué tarjeta está conectada a la red y, en caso de que una de ellas tenga un problema de red, solo tendremos que solventarlo de la manera adecuada (bien volviendo a conectar la tarjeta si ésta estaba mal conectada, o configurando correctamente su dirección IP).