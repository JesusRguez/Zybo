\section{Marco teórico}
El punto de partida del proyecto consta de una serie de tarjetas Zybo que incluyen la tecnología Zynq anteriormente citada, las cuales queremos conectar para realizar una comunicación entre ellas.

Dicha comunicación se establece para enviar un fichero entre ellas con el objetivo de recabar una cierta información.

La infraestructura a diseñar consta de los siguientes elementos:
\begin{itemize}
	\item Tarjetas Zybo Zynq-7010 (al menos 3).
	\item Un ordenador con sistema operativo Linux (Debian 10 Buster)\footnote{Es posible usar cualquier distribución de Linux.} y Windows 10.
	\item Un switch Tp-Link TL-SG1024D.
	\item Software Vivado. \textcolor{red}{Realmente yo no uso vivado para nada en mi proyecto, lo puse aquí porque Gabri sí que lo usó. ¿Qué hago, lo dejo o lo quito?}
\end{itemize}