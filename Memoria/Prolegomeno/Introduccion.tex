\chapter{Introducción}
\section{Motivación}
La motivación principal de este proyecto fue la colaboración en el proyecto de los nodos cifradores/descifradores AES basada en ApSoC.

\section{Descripción del sistema actual}
Inicialmente, se contaba con los dispositivos cifradores/descifradores AES basados en ApSoC y se detectó la necesidad de una infraestructura de red de comunicaciones entre los diferentes dispositivos. Esta infraestructura de red tendría la finalidad de conectar todos los dispositivos para que puedan añadir información a un fichero original que luego sería reenviado al terminal original (por ejemplo, un PC).

\section{Objetivos y alcance del proyecto}
\subsection{Objetivos}
El objetivo principal del proyecto es conseguir una comunicación estable y cifrada entre todos los nodos de la red.

Para cumplir con el objetivo principal, tendremos que cubrir los siguientes puntos:
\begin{itemize}
	\item Creación de rutinas que automaticen el procesado de datos.
	\item Creación de rutinas de inicio automáticas.
	\item Comprobación del estado de la red por parte de los dispositivos.
\end{itemize}

\subsection{Alcance}
Los dispositivos que se encuentren conectados a la red, deben ser capaces de comunicarse entre ellos de forma que, dado un fichero original, se descifre, se modifique su contenido, se cifre de nuevo y se envíe al siguiente nodo de la red.

\section{Organización del documento}
Este documento está organizado en función de las especificaciones expuestas para la presentación de un trabajo de fin de grado siguiendo los siguientes apartados:
\begin{enumerate}
	\item Introducción.
	\item Plan de proyecto.
	\item Análisis de requisitos.
	\item Diseño del sistema.
	\item Implementación del sistema.
	\item Pruebas del sistema.
	\item Manual de usuario.
	\item Manual de instalación.
	\item Conclusiones.
\end{enumerate}
