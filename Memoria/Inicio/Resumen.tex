\begin{center}
	\bigskip
	\bigskip
	\textbf{\huge {Resumen}}\\
	\bigskip
\end{center}
%	Qué trabajas?
	En este proyecto se ha trabajado en la creación de una estructura de red que conecta un ordenador con unos nodos cifradores/descifradores entre sí.
	
%	Para qué?
	Para tal fin, implementan un periférico de cifrado/descifrado AES implementado en su lógica programada ApSoC. Cada nodo de la red recibe información del anterior, la descifra, agrega información adicional y vuelve a cifrar el conjunto.
	
%	Cómo lo conecto físicamente?
	Para la correcta conexión de todos los dispositivos de la red se han usado cables de red UTP de categoría 5E y un switch Tp-Link TL-SG1024D.

%	Cómo se entienden ellos? Por SSH
	Un conjunto de scripts, se encarga de la recepción, el cifrado-descifrado y el envío de un fichero mediante SSH. Este proceso ha sido automatizado con el objetivo de conseguir una mayor independencia del agente humano por parte del sistema.
	
%	Cuántas?
	Aunque en las pruebas realizadas se ha usado un máximo de tres nodos cifradores/descifradores, éste proyecto está pensado para aumentar el número de dispositivos en base a las necesidades y las capacidades físicas de la red.
	
%	Basándonos en las pruebas obtenidas podemos verificar que la conexión y la automatización del proceso es exitosa.
%\end{center}

%También tenemos la opción de poner las palabras clave en este documento para que no tengan una página entera vacía solo con las palabras clave.