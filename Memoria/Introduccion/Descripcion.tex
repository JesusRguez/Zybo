\section{Descripción}
Cada uno de los nodos de la red será una tarjeta basada en la tecnología Zynq de Xilinx. Esta tecnología incluye un procesador ARM dual-core que se encargará de gestionar las
comunicaciones en la red mediante protocolo TCP/IP. Para ello, se incluirá un sistema operativo basado en Linux.

El otro elemento constituyente de Zynq es lógica programable, en la que estará implementado el periférico cifrador/descifrador, ya diseñado y verificado en el trabajo de Fin de Grado de Cristian Ambrosio Costoya.

El diseño de la infraestructura de red implica la instalación de un arranque autónomo de cada tarjeta desde una memoria SD. La interconexión física de las tarjetas y el monitor a través de
un switch mediante una topología Ethernet, siendo el número de nodos ampliable de forma dinámica y automática. La creación y ejecución de un conjunto de pruebas que permitan confirmar
el correcto funcionamiento de la red, en primera instancia, y el correcto funcionamiento del sistema de envío/recepción de datos, en segunda.