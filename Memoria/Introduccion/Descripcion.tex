\chapter{Descripción}
\section{Descripción del sistema actual}
Inicialmente, se contaba con los dispositivos cifradores/descifradores AES basados en ApSoC y se detectó la necesidad de una infraestructura de red de comunicaciones entre los diferentes dispositivos. Esta infraestructura de red tendría la finalidad de conectar todos los dispositivos para que puedan añadir información a un fichero original que luego sería reenviado al terminal original (por ejemplo, un PC).


%En este capítulo se recoge la planificación y el planteamiento de un proyecto al que hemos denominado ``\textbf{Infraestructura de red de nodos cifradores/descifradores AES basada en ApSoC}''.
%
%\section{Metodología de desarrollo}
%\textcolor{red}{No se la metodología}
%
%\section{Planificación del proyecto}
%El proyecto tendrá una duración de tres meses y se realizarán reuniones semanales con el cliente de una hora de duración como máximo.
%\newpage
%\begin{figure}[h]
%	\centering
%%	\includegraphics[scale=0.35]{Fantasy.png}
%	\caption{Diagrama de Gantt}
%	\label{Diagrama de Gantt}
%\end{figure}
%
%\section{Hitos} %Seguir poniendo los demás sprints
%\textcolor{red}{Poner los sprints}
%
%
%\section{Reuniones}
%\textcolor{red}{Poner las reuniones}
%
%\section{Recursos hardware y software}
%\textcolor{red}{Aquí poner las tarjetas vivado y demás.}
%
%\section{Costes}
%\subsection{Costes humanos}
%\textcolor{red}{Poner los costes personales}
%
%\subsection{Costes materiales}
%\textcolor{red}{Costes de las tarjetas}
%
%\section{Gestión de riesgos}
%\textcolor{red}{No se que riesgo hay}
%
