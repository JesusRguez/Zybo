\section{Conclusiones}
Después de las pruebas llevadas a cabo, podemos concluir el proyecto con un enfoque positivo. Gracias a la infraestructura de red creada, se ha conseguido con éxito una recopilación de datos totalmente anónima partiendo de un fichero inicial de datos desde el ordenador central, pasando por todos los nodos de la red y llegando, de nuevo, al ordenador central.

\section{Trabajo futuro}
Como trabajo futuro quedaría cambiar la cadena de conexiones y que, en vez de recorrer los nodos de forma secuencial, se hiciera de forma aleatoria, para que no se supiera qué nodo ha seguido a cual.

También se podría mejorar el proyecto incluyendo un módulo Wi-Fi, de modo que los nodos, no tengan que estar necesariamente conectados por cable. Esto nos daría la posibilidad de ubicar cada nodo donde quisiéramos (dentro de las capacidades físicas de la red Wi-Fi) y así poder usar dicha estructura en un entorno de IoT\footnote{Internet of Things.} para una casa o el edificio que necesitemos.