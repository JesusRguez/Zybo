\section{Comunicación y tratado de ficheros}
En este documento se desarrolla un tutorial de envío y recepción de ficheros mediante SSH entre los dispositivos de la red de manera secuencial y automática.

Para conseguir dicha finalidad, las tarjetas iniciarán automáticamente un proceso que comprobará si les envían un fichero nuevo y, en caso de que así sea, realizarán alguna modificación en él y lo enviarán a la siguiente tarjeta, o, en caso de que dicha tarjeta sea la última, al ordenador central.

Particularmente, la tarea a realizar aquí consta de la recepción de un fichero cifrado, su descifrado, modificación, cifrado y, por último, su envío hacia el siguiente dispositivo.

Se describirá una estructura de directorios que será nrecorridos por el fichero inicial en sus distintos estados. Basándosnos en dichos estados, nombraremos los directorios tal como se describen a continuación:

\begin{itemize}
	\item \texttt{~/ficheros/recibir}: Directorio donde se recibirán los ficheros.
	\item \texttt{~/ficheros/desencriptar}: Directorio donde se dejan los ficheros una vez desencriptados.
	\item \texttt{~/ficheros/trabajar}: Directorio donde se dejan los ficheros para incluir los datos y volverlo a cifrar.
	\item \texttt{~/ficheros/enviar}: Directorio donde se dejan los ficheros que están listos para ser enviados al siguiente dispositivo.
\end{itemize}

\subsection{Introducción}
Para realizar una comunicación automática y secuencial entre los distintos dispositivos, cada tarjeta Zybo Zynq 7010 contará con la siguiente estructura de directorios:\\

%Los que están comentados es porque ésta es la versión que emula el trabajo de Cristian, descomentarla para ver el proceso completo sin la emulación del trabajo de Cristian.
\texttt{
\textasciitilde/ficheros/backups/\\
%\textcolor{white}{.}\hspace{3.15cm}|\hspace{1.8cm}|ViejoDesencriptar.txt\\
\textcolor{white}{.}\hspace{3.15cm}|\hspace{1.8cm}|ViejoEnviar.txt\\
\textcolor{white}{.}\hspace{3.15cm}|\hspace{1.8cm}|ViejoRecibir.txt\\
\textcolor{white}{.}\hspace{3.15cm}|\hspace{1.8cm}|ViejoTrabajar.txt\\
\textcolor{white}{.}\hspace{3.15cm}|desencriptar/..\\
\textcolor{white}{.}\hspace{3.15cm}|enviar/..\\
\textcolor{white}{.}\hspace{3.15cm}|recibir/..\\
\textcolor{white}{.}\hspace{3.15cm}|trabajar/..\\
\textcolor{white}{.}\hspace{3.15cm}|Automatico.sh\\
\textcolor{white}{.}\hspace{3.15cm}|Borrar.sh\\
\textcolor{white}{.}\hspace{3.15cm}|Cristian.sh\\
%\textcolor{white}{.}\hspace{3.15cm}|Desencriptando.sh\\
\textcolor{white}{.}\hspace{3.15cm}|Enviando.sh\\
\textcolor{white}{.}\hspace{3.15cm}|Recibiendo.sh\\
%\textcolor{white}{.}\hspace{3.15cm}|Tabajando.sh
}

Este árbol de directorios estará en el archivo \texttt{ficheros.tar.gz} que se encontrará en el ordenador central y será distribuido a cada tarjeta mediante ssh\footnote{Recordemos que, tanto para \texttt{ssh} como para \texttt{scp}, el elemento \texttt{zyboX} es el identificador de la tarjeta Zybo con la que estamos trabajando.} con el siguiente comando:
\begin{center}
	\texttt{sshpass -p zyboX scp -o StrictHostKeyChecking=no ficheros.tar.gz zyboX@zyboX:}
\end{center}

Luego, entramos en la tarjeta mediante ssh con el siguiente comando:
\begin{center}
	\texttt{sshpass -p zyboX ssh -o StrictHostKeyChecking=no zyboX@zuboX}
\end{center}

Para ver si tenemos el archivo \texttt{ficheros.tar.gz} usamos el comando:
\begin{center}
	\texttt{ls}
\end{center}

A continuación, aplicamos el siguiente comando para descomprimir el archivo\\ \texttt{ficheros.tar.gz}:
\begin{center}
	\texttt{tar -xzvf ficheros.tar.gz}
\end{center}

Se nos creará el árbol de directorios anteriormente citado en el directorio \texttt{/home} de la tarjeta Zybo a la que hayamos accedido.

Para que la automatización del proceso se lleve a cabo correctamente, también tendremos que modificar el fichero \texttt{/etc/hosts} de la tarjeta. Para ello, enviamos el fichero con el siguiente comando:
\begin{center}
	\texttt{sshpass -p zyboX scp -o StrictHostKeyChecking=no hosts zyboX@zyboX:}
\end{center}

Y, luego, lo copiamos como super-usuario\footnote{Comando: \texttt{su}, y contraseña \texttt{root}.} en el directorio \texttt{/etc} con el siguiente comando:
\begin{center}
	\texttt{cp hosts /etc}
\end{center}

Una vez hecho esto, el proceso de automatización estaría listo para ser lanzado.

\subsection{Directorios}
En este apartado, describiremos los distintos directorios que podemos encontrar en las tarjetas Zybo una vez que se ha descomprimido el archivo \texttt{ficheros.tar.gz}.

Dicha descripción se hará siguiendo el orden que recorrerá el archivo recibido desde el ordenador central, salvo el directorio \texttt{/backups} que se describirá primero ya que no interviene de forma directa en el camino a recorrer por el archivo recibido.

\subsubsection{Directorio \texttt{/backups}}
En este directorio se guardarán los últimos estados de los directorios nombrados a continuación, generados por el comando \texttt{stat}\footnote{Para más información leer el manual del comando \texttt{stat} en este \href{https://linux.die.net/man/2/stat}{\textcolor{blue}{enlace}}.}.
\begin{itemize}
	\item \textbf{\texttt{ViejoDesencriptar.txt}:} Este fichero contendrá el estado del directorio \texttt{/desencriptar} una vez que el script \texttt{Desencriptando.sh} lo compruebe. Si no se producen cambios en dicho directorio, este fichero no se modificará.
	\item \textbf{\texttt{ViejoEnviar.txt}:} Este fichero contendrá el estado del directorio \texttt{/enviar} una vez que el script \texttt{Enviando.sh} lo compruebe. Si no se producen cambios en dicho directorio, este fichero no se modificará.
	\item \textbf{\texttt{ViejoRecibir.txt}:} Este fichero contendrá el estado del directorio \texttt{/recibir} una vez que el script \texttt{Recibiendo.sh} lo compruebe. Si no se producen cambios en dicho directorio, este fichero no se modificará.
	\item \textbf{\texttt{ViejoTrabajar.txt}:} Este fichero contendrá el estado del directorio \texttt{/trabajar} una vez que el script \texttt{Trabajando.sh} lo compruebe. Si no se producen cambios en dicho directorio, este fichero no se modificará.
\end{itemize}

\subsubsection{Directorio \texttt{/recibir}}
Este directorio contendrá los ficheros enviados por otros dispositivos mediante ssh\footnote{Para ver como enviar ficheros desde un dispositivo a otro hasta este directorio, ver el documento ``Envío y recepción de ficheros con sshpass''.}.

\subsubsection{Directorio \texttt{/desencriptar}}
Este directorio contendrá los ficheros enviados desde el directorio \texttt{/recibir} por el script \texttt{Recibiendo.sh} que comprobará el estado del directorio anterior.

\subsubsection{Directorio \texttt{/trabajar}}
Este directorio contendrá los ficheros enviados por el script \texttt{Cristian.sh} que comprobará el estado del directorio anterior.

Será donde ser realicen las acciones específicas que se contemplan en este tutorial.

\subsubsection{Directorio \texttt{/enviar}}
Este directorio contendrá los ficheros enviados por el script \texttt{Cristian.sh} que comprobará el estado del directorio anterior.

Una vez aquí, dichos ficheros estarán listos para pasar al siguiente dispositivo.



