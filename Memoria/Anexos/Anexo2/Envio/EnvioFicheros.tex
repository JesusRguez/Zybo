\section{Envío y recepción de ficheros}
Para el envío y recepción de archivos entre los distintos dispositivos, usaremos la utilidad \texttt{sshpass} \hyperlink{7}{[7]} que está diseñada para ejecutar ssh de modo no-interactivo.

Para instalar sshpass lo haremos de la siguiente forma tanto en el ordenador central como en las tarjetas zybo:
\begin{enumerate}
	\item Entramos en modo super-usuario con el comando: \texttt{su}\footnote{La contraseña será \texttt{zybomonitor} (para el ordenador central), o \texttt{root} (para las tarjetas).}.
	\item Introducimos el siguiente comando para la instalación de sshpass:
	\begin{center}
		\texttt{apt-get install sshpass}
	\end{center}
\end{enumerate}
	
\subsection{Entre ordenador y tarjeta}
Para el envío de archivos, independientemente de su extensión, mediante ssh\footnote{La opción ``\texttt{-o StrictHostKeyChecking=no}'' que encontraremos en los siguientes comandos se usa para evadir la confirmación de conexión y que la acepte inmediatamente.} desde el ordenador a las tarjetas Zybo debemos usar el siguiente comando en un terminal del ordenador ubicado en el directorio donde está el archivo que queramos enviar:
\begin{center}
	\texttt{sshpass -p zyboX scp -o StrictHostKeyChecking=no archivoLocal zyboX@zyboX:/home/zyboX/ficheros/recibir}
\end{center}
Siendo:
\begin{itemize}
%	\item \textbf{contraseña:} Será la contraseña del usuario de la tarjeta zyboX.
	\item \textbf{zyboX:} La tarjeta Zybo a la que queremos enviar el archivo\footnote{Gracias a la existencia del fichero \texttt{/etc/hosts} tenemos asociada cada tarjeta con su dirección de red. Por lo tanto, solo tenemos que poner el alias de la tarjeta para referirnos a su dirección IP.}. Por ejemplo: \texttt{zybo1}.
	\item \textbf{archivoLocal:} Nombre del archivo local que queremos enviar.
	\item \textbf{Directorio \texttt{/ficheros/recibir}:} Directorio donde se recibirán los archivos en el proyecto\footnote{Si queremos enviar un fichero a otra ubicación, solo debemos cambiar la ruta donde queremos enviarlo.}.
\end{itemize}

\newpage
\subsection{Entre tarjetas}
Para el envío de archivos entre las tarjetas Zybo tenemos dos formas, manual y automática:

\subsubsection{Manual}
Debemos conectarnos a las placas por SSH, desde el ordenador central, usando el siguiente comando:
\begin{center}
	\texttt{sshpass -p zyboX ssh -o StrictHostKeyChecking=no zyboX@zyboX}
\end{center}
Donde \texttt{X} es el identificador de la placa a la que nos queremos conectar.

Luego, nos situamos en el directorio donde se encuentra el archivo de la primera placa que queramos enviar a la segunda, y escribimos el siguiente comando:
\begin{center}
	\texttt{sshpass -p zyboX scp -o StrictHostKeyChecking=no archivoLocal zyboX@zyboX:/home/zyboX/ficheros/recibir}
\end{center}
Siendo:
\begin{itemize}
	\item \textbf{zyboX:} La tarjeta Zybo a la que queremos enviar el archivo. Por ejemplo: \texttt{zybo1}.
	\item \textbf{archivoLocal:} Nombre del archivo local que queremos enviar.
	\item \textbf{Directorio \texttt{/ficheros/recibir}:} Directorio donde se recibirán los archivos en el proyecto\footnote{Si quieremos cambiar el directorio, solo tenemos que cambiarlo al igual que en el caso ordenador-placa.}.
\end{itemize}

\subsubsection{Automática}
Usaremos el script \texttt{Enviando.sh} situado en el directorio \texttt{\textasciitilde{}/ficheros} de las placas Zybo.
Este script lo podemos encontrar en el apéndice \hyperlink{ScriptEnviando}{\texttt{Enviando.sh}}

Este script usará el mismo comando que el scp \hyperlink{7}{[7]} de forma manual, pero estará parametrizado para que lo envíe al siguiente dispositivo.