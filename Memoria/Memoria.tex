%%\documentclass[a4paper,12pt,oneside]{llncs}
\documentclass{book}
%\usepackage[right=2cm,left=3cm,top=2cm,bottom=2cm,headsep=0cm]{geometry}

%%%%%%%%%%%%%%%%%%%%%%%%%%%%%%%%%%%%%%%%%%%%%%%%%%%%%%%%%%%
%% Juego de caracteres usado en el archivo fuente: UTF-8
\usepackage{ucs}
\usepackage[utf8x]{inputenc}
%\usepackage{eurosym}

%%%%%%%%%%%%%%%%%%%%%%%%%%%%%%%%%%%%%%%%%%%%%%%%%%%%%%%%%%%
%% Juego de caracteres usado en la salida dvi
% Otra posibilidad: \usepackage{t1enc}
\usepackage[T1]{fontenc}

%%%%%%%%%%%%%%%%%%%%%%%%%%%%%%%%%%%%%%%%%%%%%%%%%%%%%%%%%%%
%% Ajusta maergenes para a4
\usepackage{a4wide}

%%%%%%%%%%%%%%%%%%%%%%%%%%%%%%%%%%%%%%%%%%%%%%%%%%%%%%%%%%%
%% Uso fuente postscript times, para que los ps y pdf queden y pequeños...
\usepackage{times}

%%%%%%%%%%%%%%%%%%%%%%%%%%%%%%%%%%%%%%%%%%%%%%%%%%%%%%%%%%%
%% Posibilidad de hipertexto (especialmente en pdf)
\usepackage{hyperref}

%%%%%%%%%%%%%%%%%%%%%%%%%%%%%%%%%%%%%%%%%%%%%%%%%%%%%%%%%%%
%% Graficos 
\usepackage{graphics,graphicx}

%%%%%%%%%%%%%%%%%%%%%%%%%%%%%%%%%%%%%%%%%%%%%%%%%%%%%%%%%%%
%% Ciertos caracteres "raros"...
\usepackage{latexsym}

%%%%%%%%%%%%%%%%%%%%%%%%%%%%%%%%%%%%%%%%%%%%%%%%%%%%%%%%%%%
%% Matematicas aun más fuertes (american math dociety)
%\usepackage{amsmath}

%%%%%%%%%%%%%%%%%%%%%%%%%%%%%%%%%%%%%%%%%%%%%%%%%%%%%%%%%%%
\usepackage{multirow} % para las tablas
%\usepackage[spanish,es-tabla]{babel}

%%%%%%%%%%%%%%%%%%%%%%%%%%%%%%%%%%%%%%%%%%%%%%%%%%%%%%%%%%%
%% Fuentes matematicas lo mas compatibles posibles con postscript (times)
%% (Esto no funciona para todos los simbolos pero reduce mucho el tamaño del
%% pdf si hay muchas matamaticas....
%\usepackage{mathptm}

%%% VARIOS:
%\usepackage{slashbox}
\usepackage{verbatim}
\usepackage{array}
\usepackage{listings}
\usepackage{multirow}
\usepackage{hhline}
\usepackage{titling}

%% MARCA DE AGUA
%% Este package de "draft copy" NO funciona con pdflatex
%%\usepackage{draftcopy}
%% Este package de "draft copy" SI funciona con pdflatex
%%%\usepackage{pdfdraftcopy}
%%%%%%%%%%%%%%%%%%%%%%%%%%%%%%%%%%%%%%%%%%%%%%%%%%%%%%%%%%%
%% Indenteacion en español...
%\usepackage[spanish]{babel}
\usepackage{Estilos/Apuntes}
\usepackage[svgnames,x11names,table]{xcolor}
\usepackage{listingsutf8}
% Para escribir código en C
% \begin{verbatim}[language=C]
% #include <stdio.h>
% int main(int argc, char* argv[]) {
% puts("Hola mundo!");
% }
% \end{verbatim}
\usepackage{hyphenat}

\newenvironment{changemargin}[2]{%
	\begin{list}{}{%
			\setlength{\topsep}{0pt}%
			\setlength{\leftmargin}{#1}%
			\setlength{\rightmargin}{#2}%
			\setlength{\listparindent}{\parindent}%
			\setlength{\itemindent}{\parindent}%
			\setlength{\parsep}{\parskip}%
		}%
		\item[]}{\end{list}}
	
\newenvironment{nota}{
	\begin{changemargin}{2em}{2em}
		\textbf{\textsc{Nota: }}
	}{
	\end{changemargin}
}


\title{\huge{Infraestructura de red de nodos cifradores/descifradores AES basada en ApSoC}}
\author{Jesús Rodríguez Heras}


%%Configuracion del paquete listings
\lstset{language=bash, numbers=left, numberstyle=\tiny, numbersep=10pt, firstnumber=1, stepnumber=1, basicstyle=\small\ttfamily, tabsize=1, extendedchars=true, inputencoding=utf8/latin1, breaklines=true}

\begin{document}
%	\maketitle

%	\begin{titlepage}
%		\centering
%		
%		{\scshape\huge Escuela Superior de Ingeniería \par}
%		\vspace{1cm}
%		{\scshape\LARGE Universidad de Cádiz\par}
%		\vspace{1cm}
%		{\scshape\Large{Stimey}\par}
%		\vspace{1cm}
%		{\Huge\bfseries Fantasy\par}
%		\vspace{1cm}
%		{\Large\itshape Luis Gutiérrez Flores\\
%			Nicolás Ruiz Requejo\\
%			Jesús Rodríguez Heras\\
%			Arantzazu Otal Alberro\\
%			Alejandro Segovia Gallardo\\
%			Alejandro José Caraballo García\\
%			Gabriel Fernando Sánchez Reina\par}
%		\vspace{2.5cm}
%		\begin{table}[htb]
%			\centering
%			\begin{tabular}{ccc}
%				\includegraphics[width=0.15\textwidth]{UCA.png}\par\vspace{1.2cm} & \includegraphics[width=0.15\textwidth]{ESI.png}\par\vspace{1.2cm} & \includegraphics[width=0.15\textwidth]{Stimey.png}\par\vspace{1.2cm}
%			\end{tabular}
%		\end{table}
%%		\vfill
%		
%		
%		
%		% Bottom of the page
%%		{\large \today\par}
%	\end{titlepage}

% Portada externa

\begin{titlepage}
	\centering
%	\includegraphics[width=.1\textwidth]{UCA.png}

\begin{table}[htb]
				\centering
				\begin{tabular}{cc}
					\includegraphics[width=0.15\textwidth]{UCA.png}\par\vspace{0.2cm} & \includegraphics[width=0.15\textwidth]{ESI.png}\par\vspace{0.2cm}
				\end{tabular}
			\end{table}
	
%	\bigskip
%	\bigskip
%	\bigskip
	
	\begin{changemargin}{3em}{3em}
		\centering
		
		{\LARGE \textsc{\nohyphens{Escuela Superior de Ingeniería}}}
		
		\bigskip
		\bigskip
		\bigskip
		\bigskip
		
		{\LARGE \nohyphens{Grado en Ingeniería Informática}}
		
		\bigskip
		\bigskip
%		\bigskip
		\bigskip
		\bigskip
		\bigskip
		
		{\LARGE \nohyphens{\textbf{Infraestructura de red de nodos cifradores/descifradores AES basada en ApSoC}}}
		
		\bigskip
		\bigskip
%		\bigskip
		\bigskip
		\bigskip
		
		{\large Curso 2019-2020}
		
		\bigskip
		\bigskip
%		\bigskip
%		\bigskip
		\bigskip
		\bigskip
		
	\end{changemargin}
	
	{\Large Jesús Rodríguez Heras} \\
	\bigskip
	\bigskip 
	\bigskip 
	{\large Puerto Real, \today}
	
\end{titlepage}

\newpage{\pagestyle{empty}\cleardoublepage}  

% Primera portada interna

{
	\thispagestyle{empty} 
	\centering
%	\includegraphics[width=.1\textwidth]{UCA.png}
\begin{table}[htb]
	\centering
	\begin{tabular}{cc}
		\includegraphics[width=0.15\textwidth]{UCA.png}\par\vspace{0.2cm} & \includegraphics[width=0.15\textwidth]{ESI.png}\par\vspace{0.2cm}
	\end{tabular}
\end{table}
	
%	\bigskip
%	\bigskip
%	\bigskip
	
	\begin{changemargin}{3em}{3em}
		
		\begin{center}
			{\LARGE \textsc{\nohyphens{Escuela Superior de Ingeniería}}}
			
			\bigskip
			\bigskip
			
			{\LARGE \nohyphens{Grado en Ingeniería Informática}}
			
			\bigskip
			\bigskip
			\bigskip
			\bigskip
			
			{\LARGE \nohyphens{\textbf{Infraestructura de red de nodos cifradores/descifradores AES basada en ApSoC}}}
			
			\bigskip
			\bigskip
			\bigskip
			\bigskip
			
		\end{center}
	\end{changemargin}
	
	\begin{flushleft}
		\Large
		
		\textsc{Departamento}: \nohyphens{Ingeniería Informática.} \\
		\textsc{Directora del proyecto}: \nohyphens{María Ángeles Cifredo Chacón.} \\
		\textsc{Codirectora del proyecto}: \nohyphens{María Mercedes Rodríguez García.} \\		
		\textsc{Autor del proyecto}: \nohyphens{Jesús Rodríguez Heras}. \\
	\end{flushleft}
	
	\bigskip
	\bigskip
	\bigskip
	
	\begin{flushright}
		\large
		Puerto Real, \today
		
		\bigskip    
		\bigskip
		\bigskip
		\bigskip
		\bigskip
		\bigskip
		\bigskip
		\bigskip
		Fdo.: Jesús Rodríguez Heras
		
	\end{flushright}
	
}

\newpage{\pagestyle{empty}\cleardoublepage}  

% Segunda portada interna

\begin{center}
	\Large{Declaración personal de auditoría}
\end{center}
Jesús Rodríguez Heras con DNI 32088516C, estudiante del título de Grado de Ingeniería Informática en la Escuela Superior de Ingeniería de la Universidad de Cádiz, como autor de este documento académico titulado ``Infraestructura de red de nodos cifradores/descifradores AES basada en ApSoC'' y presentado como Trabajo Final de Grado

DECLARO QUE:

Es un trabajo original, que no copio ni utilizo parte de obra alguna sin mencionar de forma
clara y precisa su origen tanto en el cuerpo del texto como en su bibliografía y que no empleo datos de terceros sin la debida autorización, de acuerdo con la legislación vigente. Asimismo, declaro que soy plenamente consciente de que no respetar esta obligación podrá implicar la aplicación de sanciones académicas, sin perjuicio de otras actuaciones que pudieran iniciarse.

En Puerto Real, a \today.

\bigskip
\bigskip
\bigskip
\bigskip
\bigskip
Fdo.: Jesús Rodríguez Heras
	
%	\thispagestyle{empty}
	\newpage{\pagestyle{empty}\cleardoublepage}  
	
%	\begin{center}
	\bigskip
	\bigskip
	\textbf{\huge {Resumen}}\\
	\bigskip
\end{center}

	En este proyecto se ha trabajado en el diseño de una estructura de red que conecta un ordenador con unos nodos cifradores/descifradores basados en tecnología programable ApSoC.Para tal fin, los nodos implementan un periférico hardware de cifrado/descifrado AES-128 implementado en la lógica programable ApSoC. Cada nodo de la red recibe información del nodo anterior, la descifra, agrega información  vuelve a cifrar el conjunto para enviarlo al siguiente nodo.
	
	Para la conexión de todos los dispositivos de red se han usado cables de red UTP de categoría 5E y un switch TP-Link TL-SG1024D. Se ha desarrollado un conjunto de scripts que automatizan el proceso al completo, encargándose por tanto de la recepción, cifrado/descifrado y envío del conjunto de información mediante el protocolo de comunicación SSH. Este proceso ha sido automatizado con el objetivo de conseguir una mayor independencia del agente humano por parte del sistema.
	
	La red montada para el proceso de test y verificación del sistema, usa un total de tres nodos, sin embargo, el diseño permite aumentar el número de dispositivos en base a las necesidades y capacidades físicas de la red.

	\begin{center}
	\bigskip
	\bigskip
	\textbf{\huge {Agradecimientos}}\\
	\bigskip
	Me gustaría mostrar mis agradecimientos a la gente.
\end{center}
	\newpage{\pagestyle{empty}\cleardoublepage}  
	\begin{center}
	\bigskip
	\bigskip
	\textbf{\huge {Resumen}}\\
	\bigskip
\end{center}

	En este proyecto se ha trabajado en el diseño de una estructura de red que conecta un ordenador con unos nodos cifradores/descifradores basados en tecnología programable ApSoC.Para tal fin, los nodos implementan un periférico hardware de cifrado/descifrado AES-128 implementado en la lógica programable ApSoC. Cada nodo de la red recibe información del nodo anterior, la descifra, agrega información  vuelve a cifrar el conjunto para enviarlo al siguiente nodo.
	
	Para la conexión de todos los dispositivos de red se han usado cables de red UTP de categoría 5E y un switch TP-Link TL-SG1024D. Se ha desarrollado un conjunto de scripts que automatizan el proceso al completo, encargándose por tanto de la recepción, cifrado/descifrado y envío del conjunto de información mediante el protocolo de comunicación SSH. Este proceso ha sido automatizado con el objetivo de conseguir una mayor independencia del agente humano por parte del sistema.
	
	La red montada para el proceso de test y verificación del sistema, usa un total de tres nodos, sin embargo, el diseño permite aumentar el número de dispositivos en base a las necesidades y capacidades físicas de la red.
	\newpage{\pagestyle{empty}\cleardoublepage}  
	\begin{center}
	\bigskip
	\bigskip
	\textbf{\huge {Palabras clave}}\\
	\bigskip
	Red, Infraestructura, Zybo, Conexión.
\end{center}
	
	\tableofcontents
	\newpage
	\listoffigures
	\newpage
	\listoftables
	\newpage
	
	\part{Introducción}
	\chapter{Objetivos}
\section{Objetivos}
%El objetivo principal de este proyecto es la creación de una estructura de red funcional para las interconexión entre nodos cifradores/descifradores AES basada en ApSoC.

El objetivo principal del proyecto es conseguir una comunicación estable y cifrada entre todos los nodos de la red.

Para cumplir con el objetivo principal, tendremos que cubrir los siguientes puntos:
\begin{itemize}
	\item Creación de rutinas que automaticen el procesado de datos.
	\item Creación de rutinas de inicio automáticas.
	\item Comprobación del estado de la red por parte de los dispositivos.
\end{itemize}

	\chapter{Descripción}
Cada uno de los nodos de la red será una tarjeta basada en la tecnología Zynq de Xilinx. Esta tecnología incluye un procesador ARM dual-core que se encargará de gestionar las
comunicaciones en la red mediante protocolo TCP/IP. El otro elemento constituyente de Zynq es lógica programable, en la que estará implementado el periféricos o IP, ya diseñado y
verificado, para el cifrado/descifrado AES. Se evaluará la posibilidad de que cada tarjeta incluya solo lo necesario para contar con comunicación TCP/IP o bien un sistema operativo
basado en Linux.

El diseño de la infraestructura de red implica la instalación de un arranque autónomo de cada tarjeta desde memoria SD. La interconexión física de las tarjetas y el monitor a través de
un switch mediante topología Ethernet, siendo el número de nodos ampliable de forma dinámica y automática. La creación y ejecución de un conjunto de pruebas que permitan confirmar
el correcto funcionamiento de la red, en primera instancia, y el correcto funcionamiento del sistema de envío/recepción de datos, en segunda.
	\section{Alcance}
El trabajo incluirá:
\begin{itemize}
	\item Instalación física del ordenador personal que actuará como monitor.
	\item Creación de una imagen de arranque en tarjeta SD para las placas que formarán parte de la red. El arranque incluirá el bitstream necesario para configurar la lógica programable de Zynq con el IP AES core, así como el sistema operativo y los scripts.
	\item Instalación física y configuración de cada tarjeta en la red, y configuración del switch.
	\item Creación de los scripts necesarios para que cada nodo/tarjeta sea capaz de:
	\begin{itemize}
		\item Recibir datos.
		\item Descifre datos.
		\item Añada datos.
		\item Cifre datos.
		\item Envíe el conjunto datos a otro nodo.
	\end{itemize}
	\item Creación y ejecución de los tests que permitan comprobar el correcto funcionamiento de la infraestructura.
	\item Preparación del fichero de datos inicial en el monitor.
	\item Creación y ejecución de los tests que permitan comprobar el correcto funcionamiento de la transferencia, cifrado/descifrado y agregación de datos.
\end{itemize}
	
	\part{Metodología}
	\section{Marco teórico}
El punto de partida del proyecto consta de una serie de tarjetas Zybo \hyperlink{3}{[3]} que incluyen la tecnología Zynq anteriormente citada, las cuales actuarán como nodos y que queremos conectar para realizar una comunicación entre ellas.

Dicha comunicación se establece para enviar un fichero entre ellas con el objetivo de recolectar una cierta información de cada una de ellas y enviarla al ordenador central.

\subsection{Tarjetas Zybo}
\textcolor{red}{Ampliar este subapartado buscando las referencias oficiales.}
	\section{Tecnologías a utilizar}
\subsection{Diseño de la arquitectura}
La arquitectura consta de una red privada\footnote{La red será cableada mediante cables UTP de categoría 5e.}, interconectada mediante un switch, las tarjetas Zybo y un ordenador que será denominado "monitor" ya que será el encargado de enviar los ficheros iniciales y recibir los ficheros con los datos finales.

Los datos manejados por las tarjetas Zybo estarán cifrados usando el cifrado AES. Para ello se ha empleado un IP\footnote{Intelectual Property.} diseñado y verificado en otro TFG realizado por Cristian Ambrosio Costoya.

\subsection{Diseño de componentes}
Los componentes principales de este proyecto son las tarjetas Zybo Zynq 7010 que tienen las siguientes características:
\begin{itemize}
	\item Procesador Cortex-A9 doble núcleo a 650 MHz.
	\item Memoria DDR3 con 8 canales de DMA\footnote{Acceso Directo a Memoria: Permite a cierto tipo de componentes de una computadora acceder a la memoria del sistema para leer o escribir independientemente de la unidad central de procesamiento (CPU) principal.}.
	\item Controladores de periféricos de gran ancho de banda: 1Gb Ethernet, USB 2.0.
	\item Controladores de periféricos de bajo ancho de banda: SPI, UART, CAN, I$^2$C.
	\item Ranura MicroSD (compatible con el sistema de archivos Linux).
	\item Lógica reprogramable equivalente a Artix-7 FPGA:
	\begin{itemize}
		\item 4.400 segmentos lógicos, cada uno con cuatro LUT de 6 entradas y 8 flip-flops.
		\item 512 MB x32 DDR3 con ancho de banda de 1050 Mbps.
		\item Dos cristales de administración de reloj, cada uno con un bucle de fase bloqueada (PLL) y un administrador de reloj de modo mixto (MMCM).
		\item Procesadores digitales de señales de 80 componentes.
		\item Velocidades de reloj interno que exceden los 450MHz.
		\item Convertidor analógico a digital en chip (XADC).
	\end{itemize}
\end{itemize}

Cada tarjeta Zybo incluye una instalación de un sistema operativo Linux para facilitar la gestión de los ficheros. El sistema operativo usado en el proyecto es Xillinux el cual ha sido desarrollado por Xilinx e irá instalado en la tarjeta micro SD de cada tarjeta Zybo.

Para la instalación del sistema operativo podemos ver el apéndice \hyperlink{InstalacionLinux}{Instalación de Linux en SD para tarjetas Zybo.}

%\textcolor{red}{No se si voy a usar el SO de Gabri o si voy a implementar lo de gabri en el mío, la verdad es que me da igual una cosa que otra. El problema está en que le he preguntado si tiene el sistema operativo por ahí y cree que lo ha borrado así que no se cómo hacer esto del sistema operativo, porque debe de tener tanto lo de Gabri como lo de Cristian para que funcione todo correctamente. Use el que use, lo tengo que detallar aquí.}
	\section{Análisis del sistema}
Una vez que los dispositivos están conectados y se ha creado la infraestructura de red (ver anexo \hyperlink{CreacionInfraestructura}{Creación de una infraestructura de red con tarjetas Zybo}), el sistema es capaz de cifrar los datos, enviarlos por el conjunto de placas que forman la red, descifrar los datos y volverlos a enviar. De esta forma, se consigue que la recolecta de datos sea totalmente cifrada, ya que las comunicaciones están cifradas gracias al protocolo AES y al protocolo SSH mediante el cual se envían y reciben los datos.

\subsection{Hardware}
%Requisitos de la red en si, explicando lo que tendrá que haber y como deberá conectarse, sin decir todavía cómo.
La red estará formada por las tarjetas Zybo, el ordenador central (monitor) y el switch. Cada tarjeta debe tener instalado un sistema operativo Linux para permitir la gestión de ficheros y el uso del protocolo SSH, que será usado en el envío y recepción de ficheros de datos.

\subsection{Software}
%Hablará de los test que habrá que diseñar y luego de los scripts que deberán automatizar el funcionamiento del sistema.
Una vez que se ha montado la red completa, debemos probar que las conexiones entre las tarjetas y el ordenador monitor están habilitadas. Para ello, se ha diseñado un pequeño script con un test de prueba de conectividad (ver anexo \hyperlink{TestConexion}{Test de interconexión de red Zybo}), el cual nos dirá qué tarjeta está conectada a la red y, en caso de que una de ellas tenga un problema de red, solo tendremos que solventarlo de la manera adecuada (bien volviendo a conectar la tarjeta si ésta estaba mal conectada, o configurando correctamente su dirección IP).
	\chapter{Diseño y desarrollo}

\section{Hardware}


\section{Software}

	\section{Pruebas del sistema}
%Aquí describimos los scripts para hacer pruebas e incluir las pruebas que hice para ver su funcionamiento.
\subsection{Hardware}
%\textcolor{red}{Prueba de la infraestructura: Lanzar el script Inicio.sh para que se vea que todas las tarjetas dan ping.}
Para realizar la prueba de la arquitectura de red, lanzaremos el script \hyperlink{ScriptConexion}{\texttt{Inicio.sh}}. Este script, nos dirá qué tarjeta está conectada o desconectada de la red. Podemos ver su ejecución en la Figura \ref{Prueba de Inicio.sh}.

\begin{figure}[h]
	\centering
	\includegraphics[scale=0.5]{Metodologia/Pruebas/Prueba_Inicio_sh.png}
	\caption{Prueba de \texttt{Inicio.sh}}
	\label{Prueba de Inicio.sh}
\end{figure}

\subsection{Software}
%\textcolor{red}{Prueba de funcionamiento de la recolección de datos: Una prueba de todo funcionando.}
Para realizar las pruebas de funcionamiento, basta con alimentar los nodos participantes en la cadena. En ese momento se ejecuta automáticamente los scripts descritos en el \hyperlink{Scripts}{Apéndice B}. 

Para el correcto funcionamiento de estos scripts, se requiere que el monitor central disponga del fichero inicial de datos creado por el usuario y que dicho usuario lo envíe al primer nodo de la red.

Hecho esto, sucede lo siguiente:
\begin{enumerate}
	\item Envío del fichero inicial a la primera tarjeta: Esto lo haremos gracias a la herramienta \texttt{sshpass} de Linux con la siguiente orden:
	\begin{center}
		\texttt{sshpass -p zyboX scp -o StrictHostKeyChecking=no archivoLocal zyboX@zyboX:/home/zyboX/ficheros/recibir}
	\end{center}
	Este proceso (Figura \ref{Fichero inicial en el ordenador central y envío al primer nodo}) lo podemos ver más detalladamente en el \hyperlink{EnvioRecepcionFicheros}{Apéndce A.4}.
	\begin{figure}[h]
		\centering
		\includegraphics[scale=0.5]{Metodologia/Pruebas/Fichero_inicial_en_PC.png}
		\caption{Fichero inicial en el ordenador central y envío al primer nodo}
		\label{Fichero inicial en el ordenador central y envío al primer nodo}
	\end{figure}
\newpage
	\item Recepción del fichero en las tarjetas: Será el script \hyperlink{ScriptRecibiendo}{\texttt{Recibiendo.sh}} el encargado de comprobar la llegada del fichero y actuar en consecuencia cambiándolo de directorio.

	\item Modificación del fichero: El script \hyperlink{ScriptCristian}{\texttt{Cristian.sh}} será el encargado de abrir el fichero, modificarlo en cada una de las tarjetas y dejarlo preparado para su envío al siguiente nodo de la red. Para comprobar que los resultados de este script son correctos, podemos usar los comandos que vemos en la Figura \ref{Estado del fichero después de su modificación en Zybo1}.
	\begin{figure}[h]
		\centering
		\includegraphics[scale=0.5]{Metodologia/Pruebas/Fichero_en_Zybo1.png}
		\caption{Estado del fichero después de su modificación en Zybo1}
		\label{Estado del fichero después de su modificación en Zybo1}
	\end{figure}
\newpage
	\item Envío del fichero hacia el siguiente nodo: Podemos distinguir dos tipos de envío del mismo fichero. Ambos llevados a cabo por el script \hyperlink{ScriptEnviando}{Enviando.sh}:
	\begin{enumerate}
		\item Tarjeta-Tarjeta: Esta opción se dará cuando la tarjeta actual detecte que la siguiente tarjeta está conectada.
		\item Tarjeta-Ordenador: Esta opción se dará cuando la tarjeta no detecte a la siguiente tarjeta. Entonces, enviará el fichero de datos, de vuelta al ordenador central.
	\end{enumerate}

%\newpage
	\item Recepción del fichero en el ordenador central: El fichero final de datos, será recibido en el directorio\\ \texttt{/home/jesus/Vídeos} del ordenador central y contendrá los datos añadidos por todos los nodos de la red. Podemos ver el fichero final en la Figura \ref{Fichero final en el ordenador central}.
	\newpage
	\begin{figure}[h]
		\centering
		\includegraphics[scale=0.5]{Metodologia/Pruebas/Fichero_final_en_PC.png}
		\caption{Fichero final en el ordenador central}
		\label{Fichero final en el ordenador central}
	\end{figure}
\end{enumerate}

	\part{Conclusiones y trabajo futuro}
	\section{Conclusiones}
Después de las pruebas llevadas a cabo, podemos concluir el proyecto con un enfoque positivo. Gracias a la infraestructura de red creada, se ha conseguido con éxito una recopilación de datos totalmente anónima partiendo de un fichero inicial de datos desde el ordenador central, pasando por todos los nodos de la red y llegando, de nuevo, al ordenador central.

\section{Trabajo futuro}
Como trabajo futuro quedaría cambiar la cadena de conexiones y que, en vez de recorrer los nodos de forma secuencial, se hiciera de forma aleatoria, para que no se supiera qué nodo ha seguido a cual.

También se podría mejorar el proyecto incluyendo un módulo Wi-Fi, de modo que los nodos, no tengan que estar necesariamente conectados por cable. Esto nos daría la posibilidad de ubicar cada nodo donde quisiéramos (dentro de las capacidades físicas de la red Wi-Fi) y así poder usar dicha estructura en un entorno de IoT\footnote{Internet of Things.} para una casa o el edificio que necesitemos.
	
	\part{Referencias/Bibliografía}
	\textcolor{red}{Aquí empezaré a poner la bibliografía que constará de los foros y las páginas web que se han ido consultando a la hora de realizar el proyecto.}
	
	\part{Anexos técnicos}
	\appendix
	\chapter{Datos técnicos}
\textcolor{red}{Aquí se pueden poner las especificaciones de los dispositivos involucrados en el proyecto (ordenador, placas y switch).}
	\chapter{Datos técnicos}
Aquí puedo poner algún dato técnico a tener en cuenta en el proyecto.
	\chapter{Códigos}
Aquí puedo incrustar tal cual los scripts del proyecto.
	
\end{document}
