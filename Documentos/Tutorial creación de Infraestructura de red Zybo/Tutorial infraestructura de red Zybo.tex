%%\documentclass[a4paper,12pt,oneside]{llncs}
\documentclass[12pt,letterpaper]{article}
\usepackage[right=2cm,left=3cm,top=2cm,bottom=2cm,headsep=0cm]{geometry}

%%%%%%%%%%%%%%%%%%%%%%%%%%%%%%%%%%%%%%%%%%%%%%%%%%%%%%%%%%%
%% Juego de caracteres usado en el archivo fuente: UTF-8
\usepackage{ucs}
\usepackage[utf8x]{inputenc}

%%%%%%%%%%%%%%%%%%%%%%%%%%%%%%%%%%%%%%%%%%%%%%%%%%%%%%%%%%%
%% Juego de caracteres usado en la salida dvi
%% Otra posibilidad: \usepackage{t1enc}
\usepackage[T1]{fontenc}

%%%%%%%%%%%%%%%%%%%%%%%%%%%%%%%%%%%%%%%%%%%%%%%%%%%%%%%%%%%
%% Ajusta maergenes para a4
%\usepackage{a4wide}

%%%%%%%%%%%%%%%%%%%%%%%%%%%%%%%%%%%%%%%%%%%%%%%%%%%%%%%%%%%
%% Uso fuente postscript times, para que los ps y pdf queden y pequeños...
\usepackage{times}

%%%%%%%%%%%%%%%%%%%%%%%%%%%%%%%%%%%%%%%%%%%%%%%%%%%%%%%%%%%
%% Posibilidad de hipertexto (especialmente en pdf)
%\usepackage{hyperref}
\usepackage[bookmarks = true, colorlinks=true, linkcolor = black, citecolor = black, menucolor = black, urlcolor = black]{hyperref}

%%%%%%%%%%%%%%%%%%%%%%%%%%%%%%%%%%%%%%%%%%%%%%%%%%%%%%%%%%%
%% Graficos 
\usepackage{graphics,graphicx}

%%%%%%%%%%%%%%%%%%%%%%%%%%%%%%%%%%%%%%%%%%%%%%%%%%%%%%%%%%%
%% Ciertos caracteres "raros"...
\usepackage{latexsym}

%%%%%%%%%%%%%%%%%%%%%%%%%%%%%%%%%%%%%%%%%%%%%%%%%%%%%%%%%%%
%% Matematicas aun más fuertes (american math dociety)
\usepackage{amsmath}

%%%%%%%%%%%%%%%%%%%%%%%%%%%%%%%%%%%%%%%%%%%%%%%%%%%%%%%%%%%
\usepackage{multirow} % para las tablas
\usepackage[spanish,es-tabla]{babel}

%%%%%%%%%%%%%%%%%%%%%%%%%%%%%%%%%%%%%%%%%%%%%%%%%%%%%%%%%%%
%% Fuentes matematicas lo mas compatibles posibles con postscript (times)
%% (Esto no funciona para todos los simbolos pero reduce mucho el tamaño del
%% pdf si hay muchas matamaticas....
\usepackage{mathptm}

%%% VARIOS:
%\usepackage{slashbox}
\usepackage{verbatim}
\usepackage{array}
\usepackage{listings}
\usepackage{multirow}

%% MARCA DE AGUA
%% Este package de "draft copy" NO funciona con pdflatex
%%\usepackage{draftcopy}
%% Este package de "draft copy" SI funciona con pdflatex
%%%\usepackage{pdfdraftcopy}
%%%%%%%%%%%%%%%%%%%%%%%%%%%%%%%%%%%%%%%%%%%%%%%%%%%%%%%%%%%
%% Indenteacion en español...
\usepackage[spanish]{babel}
\usepackage{pdfpages}

\usepackage{listings}
% Para escribir código en C
% \begin{lstlisting}[language=C]
% #include <stdio.h>
% int main(int argc, char* argv[]) {
% puts("Hola mundo!");
% }
% \end{lstlisting}


\title{Creación de una infraestructura de red de placas Zybo}
\author{Jesús Rodríguez Heras}

\begin{document}
	
	\maketitle
	\begin{abstract} %Poner esto en todas las prácticas de PCTR
		\begin{center}
			En este documento se desarrolla la creación de la infraestructura de red física de placas Zybo, un ordenador y un switch.
		\end{center}
	\end{abstract}
	\thispagestyle{empty}
	\newpage
	
	\tableofcontents
	\newpage
	
	%%\listoftables
	%%\newpage
	
	%%\listoffigures
	%%\newpage
	
	%%%% REAL WORK BEGINS HERE:
	
	%%Configuracion del paquete listings
	\lstset{language=bash, numbers=left, numberstyle=\tiny, numbersep=10pt, firstnumber=1, stepnumber=1, basicstyle=\small\ttfamily, tabsize=1, extendedchars=true, inputencoding=latin1}


\section{Material necesario}
Para la creación de la infraestructura de red física de placas Zybo contaremos con el siguiente material:
\begin{itemize}
	\item Placas Zybo Zynq-7010.
	\item Un ordenador con sistema operativo Linux (Debian 9 Stretch)\footnote{También es posible usar cualquier otra distribución de Linux.} y Windows 7.
	\item Un switch tp-link modelo TL-SG1024D.
	\item Software Vivado.
\end{itemize}


\subsection{Placas Zybo Zynq-7000}
\begin{figure}[h]
	\centering
	\includegraphics[scale=0.5]{zybo.jpg}
	\caption{Placa Zybo Zynq 7010}
	\label{Placa Zybo}
\end{figure}
Para este proyecto necesitaremos poder programar la FPGA integrada en la placa desde la tarjeta SD de memoria. Para ello se va a preparar una imagen para que el procesador ARM integrado en la placa arranque desde la tarjeta SD y pueda programar la FPGA. El sistema operativo elegido es Xilinux\footnote{Más información en: \url{http://xillybus.com/xillinux}.}.

Las placas Zybo Zynq 7010 tienen tres posibles modos de arranque que podemos seleccionar con el jumper JP5: QSPI, SD, JTAG. En este proyecto, el sistema operativo estará en la tarjeta SD, por lo tanto, tendremos que cambiar el jumper JP5 (situado arriba a la derecha) a la posición ``SD''\footnote{Dicho jumper está identificado con el número 21 en la siguiente página.}.
\newpage
\begin{figure}[h]
	\centering
	\includegraphics[scale=0.7]{Datasheet.pdf}
	\caption{Diagrama de Zybo Zynq 7010 substraído del manual de referencias}
	\label{Datasheet}
\end{figure}
\noindent
\url{https://www.xilinx.com/support/documentation/university/XUP\%20Boards/XUPZYBO/documentation/ZYBO_RM_B_V6.pdf}

\newpage
\subsection{Sistemas operativos}
El ordenador usado en el proyecto tendrá dos sistemas operativos.
\begin{itemize}
	\item \textbf{Debian 9 Stretch:} Este sistema operativo tendrá un usuario llamado \texttt{zybo} y su contraseña será \texttt{zybomonitor}. La contraseña para los permisos de super-usuario también será \texttt{zybomonitor}. En este sistema operativo se realizará la compilación del sistema operativo Xilinux\footnote{Más información en: \url{http://xillybus.com/xillinux}.} de las tarjetas Zybo y la programación del bitstream con el software Vivado.
	\item \textbf{Windows 7:} También tendrá la capacidad de programar la FPGA de la tarjeta usando el software Vivado..
\end{itemize}

\subsection{Software}
\begin{itemize}
	\item \textbf{Vivado:} Versión 2018.2 instalado en los sistemas operativos anteriormente mencionados.
\end{itemize}

\subsection{Switch}
El switch usado en este proyecto es el tp-link TL-SG1024D que cuenta con 24 puertos con tecnología Gigabit y conectores RJ-45. También cuenta con interfaz accesible para su configuración.


\section{Pasos para el montaje de la infraestructura}
\subsection{Preparación de la tarjeta SD para arrancar la placa con Linux}

\textcolor{red}{A la espera de que Gabri se decida por qué sistema operativo va a elegir, si el que ya hemos encontrado funcional o el de la guía de GitHub que no tiene pinta de tirar demasiado bien.}


\subsection{Creación de usuario en las placas}
Llegados a este paso las tarjetas ya tienen su sistema operativo instalado en la tarjeta SD y el jumper JP5 está en la posición ``SD'' para que la placa arranque desde dicha tarjeta.

Si es la primera vez que encendemos las placas, solo estará creado el usuario \texttt{root} con contraseña \texttt{root} (viene por defecto), por lo que necesitamos crear un usuario para cada placa.

Conectamos las placas al ordenador mediante la interfaz serie (USB), las encendemos e iniciamos un terminal serie (PuTTY) en el ordenador\footnote{Si lo hacemos en Linux usar el puerto \texttt{ttyUSB1} con una velocidad de 115200.}.

A continuación, iniciamos sesión como root e introducimos el siguiente comando:
\begin{center}
	\texttt{adduser zyboX}
\\	\textcolor{red}{Falta ponerlo mejor porque necesito probarlo en las placas de nuevo para detallarlo mejor.}
\end{center}
Donde \texttt{X} es el identificador de la placa con la que estamos trabajando.

\subsection{Asignar direcciones IP}
Para asignarles una dirección IP a los dispositivos debemos acceder al fichero\\ \texttt{/etc/network/interfaces}. Para ello usaremos el editor \texttt{vi} que es el que trae Xilinux por defecto. Accedemos a dicho fichero con el comando:
\begin{center}
	\texttt{sudo vi /etc/network/interfaces}
\end{center}
\textcolor{red}{Me falta la imagen porque también necesito estar en el laboratorio}

Localizamos la interfaz de red de los dispositivos y establecemos la dirección IP siguiendo la siguiente tabla:

\begin{table}[h]
	\centering
	\begin{tabular}{|c|c|}
		\hline
		\textbf{Dispositivo} & \textbf{Dirección IP} \\ \hline
		Monitor & 192.168.1.1 \\ \hline
		Zybo1 & 192.168.1.2 \\ \hline
		Zybo2 & 192.168.1.3 \\ \hline
		Zybo3 & 192.168.1.4 \\ \hline
		Zybo4 & 192.168.1.5 \\ \hline
	\end{tabular}
\caption{Direcciones IP de las placas}
\label{Direcciones}
\end{table}

Las tarjetas estarán identificadas como ZyboX (siendo ``X'' el identificador de la placa con la que estamos trabajando) y el ordenador se identificará como ``Monitor''.

\subsection{Conexión al switch de las placas Zybo}
\subsubsection{Acceder a la interfaz web del switch}
\textcolor{red}{No se ha probado y esto es mejor que lo hablemos personalmente porque es bastante más fácil que describirlo, simplemente con la conexión del nuevo dispositivo, ya queda todo hecho, no hace falta configurar nada cuando se crea la red (se crea al enchufar los dispositivos) ni hay que modificar nada cuando se quieran añadir nuevas placas, solo con añadirlas, ya funcionan. Lo único que quedaría por redactar sería el acceso a la interfaz web del switch, que eso lo hago en el laboratorio.}

Una vez tengamos los dispositivos identificados tenemos que conectarlos al switch\footnote{Podemos conectar los dispositivos al puerto del switch que queramos debido a que se encargará de ir rellenando su tabla CAM con las direcciones de los dispositivos que tiene conectados.}. Para probar la conectividad entre todos los dispositivos tendremos que ejecutar el test de interconexión de red.

\textcolor{red}{En cuanto al test de interconexión está en el tutorial de interconexión de red Zybo. Eso no debería ir aquí, sino en el otro tutorial.}
\end{document}