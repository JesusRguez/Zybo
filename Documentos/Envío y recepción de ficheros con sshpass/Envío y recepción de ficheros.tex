%%\documentclass[a4paper,12pt,oneside]{llncs}
\documentclass[12pt,letterpaper]{article}
\usepackage[right=2cm,left=3cm,top=2cm,bottom=2cm,headsep=0cm]{geometry}

%%%%%%%%%%%%%%%%%%%%%%%%%%%%%%%%%%%%%%%%%%%%%%%%%%%%%%%%%%%
%% Juego de caracteres usado en el archivo fuente: UTF-8
\usepackage{ucs}
\usepackage[utf8x]{inputenc}

%%%%%%%%%%%%%%%%%%%%%%%%%%%%%%%%%%%%%%%%%%%%%%%%%%%%%%%%%%%
%% Juego de caracteres usado en la salida dvi
%% Otra posibilidad: \usepackage{t1enc}
\usepackage[T1]{fontenc}

%%%%%%%%%%%%%%%%%%%%%%%%%%%%%%%%%%%%%%%%%%%%%%%%%%%%%%%%%%%
%% Ajusta maergenes para a4
%\usepackage{a4wide}

%%%%%%%%%%%%%%%%%%%%%%%%%%%%%%%%%%%%%%%%%%%%%%%%%%%%%%%%%%%
%% Uso fuente postscript times, para que los ps y pdf queden y pequeños...
\usepackage{times}

%%%%%%%%%%%%%%%%%%%%%%%%%%%%%%%%%%%%%%%%%%%%%%%%%%%%%%%%%%%
%% Posibilidad de hipertexto (especialmente en pdf)
%\usepackage{hyperref}
\usepackage[bookmarks = true, colorlinks=true, linkcolor = black, citecolor = black, menucolor = black, urlcolor = black]{hyperref}

%%%%%%%%%%%%%%%%%%%%%%%%%%%%%%%%%%%%%%%%%%%%%%%%%%%%%%%%%%%
%% Graficos 
\usepackage{graphics,graphicx}

%%%%%%%%%%%%%%%%%%%%%%%%%%%%%%%%%%%%%%%%%%%%%%%%%%%%%%%%%%%
%% Ciertos caracteres "raros"...
\usepackage{latexsym}

%%%%%%%%%%%%%%%%%%%%%%%%%%%%%%%%%%%%%%%%%%%%%%%%%%%%%%%%%%%
%% Matematicas aun más fuertes (american math dociety)
\usepackage{amsmath}

%%%%%%%%%%%%%%%%%%%%%%%%%%%%%%%%%%%%%%%%%%%%%%%%%%%%%%%%%%%
\usepackage{multirow} % para las tablas
\usepackage[spanish,es-tabla]{babel}

%%%%%%%%%%%%%%%%%%%%%%%%%%%%%%%%%%%%%%%%%%%%%%%%%%%%%%%%%%%
%% Fuentes matematicas lo mas compatibles posibles con postscript (times)
%% (Esto no funciona para todos los simbolos pero reduce mucho el tamaño del
%% pdf si hay muchas matamaticas....
\usepackage{mathptm}

%%% VARIOS:
%\usepackage{slashbox}
\usepackage{verbatim}
\usepackage{array}
\usepackage{listings}
\usepackage{multirow}

%% MARCA DE AGUA
%% Este package de "draft copy" NO funciona con pdflatex
%%\usepackage{draftcopy}
%% Este package de "draft copy" SI funciona con pdflatex
%%%\usepackage{pdfdraftcopy}
%%%%%%%%%%%%%%%%%%%%%%%%%%%%%%%%%%%%%%%%%%%%%%%%%%%%%%%%%%%
%% Indenteacion en español...
\usepackage[spanish]{babel}

\usepackage{listings}
% Para escribir código en C
% \begin{lstlisting}[language=C]
% #include <stdio.h>
% int main(int argc, char* argv[]) {
% puts("Hola mundo!");
% }
% \end{lstlisting}


\title{Envío y recepción de ficheros}
\author{Jesús Rodríguez Heras}

\begin{document}
	
	\maketitle
	\begin{abstract} %Poner esto en todas las prácticas de PCTR
		\begin{center}
			En este documento se desarrolla un tutorial de envío y recepción de ficheros mediante SSH entre las placas Zybo y el ordenador.
		\end{center}
	\end{abstract}
	\thispagestyle{empty}
	\newpage
	
	\tableofcontents
	\newpage
	
	%%\listoftables
	%%\newpage
	
	%%\listoffigures
	%%\newpage
	
	%%%% REAL WORK BEGINS HERE:
	
	%%Configuracion del paquete listings
	\lstset{language=bash, numbers=left, numberstyle=\tiny, numbersep=10pt, firstnumber=1, stepnumber=1, basicstyle=\small\ttfamily, tabsize=1, extendedchars=true, inputencoding=latin1}
	

\section{SSHPASS}
Para el envío y recepción de archivos entre los distintos dispositivos, usaremos la utilidad ``\texttt{sshpass}'' que está diseñada para ejecutar ssh de modo no-interactivo.	
	
\section{Entre ordenador y placa}
Para el envío de archivos\footnote{Podremos enviar cualquier tipo de archivo independientemente de su extensión} desde el ordenador a las tarjetas Zybo debemos usar el siguiente comando en un terminal del ordenador ubicado en el directorio donde está el archivo que queramos enviar:
\begin{center}
	\texttt{sshpass -p zyboX scp -o StrictHostKeyChecking=no archivoLocal zyboX@zyboX:/home/zyboX/ficheros/recibir}
\end{center}
Siendo:
\begin{itemize}
%	\item \textbf{contraseña:} Será la contraseña del usuario de la tarjeta zyboX.
	\item \textbf{zyboX:} La tarjeta Zybo a la que queremos enviar el archivo\footnote{Gracias a la existencia del fichero \texttt{/etc/hosts} tenemos asociada cada tarjeta con su dirección de red. Por lo tanto, solo tenemos que poner el alias de la tarjeta para referirnos a su dirección IP}. Por ejemplo: \texttt{zybo1}.
	\item \textbf{archivoLocal:} Nombre del archivo local que queremos enviar.
	\item \textbf{Directorio \texttt{/ficheros/recibir}:} Directorio donde se recibirán los archivos en el proyecto de anonimización\footnote{Si queremos enviar un fichero a otra ubicación, solo debemos cambiar la ruta donde queremos copiarlo.}.
\end{itemize}

\section{Entre placas}
Para el envío de archivos entre las tarjetas Zybo tenemos dos formas, manual y automático:

\subsection{Manual}
Debemos conectarnos a las placas por SSH, desde el ordenador central, usando el siguiente comando:
\begin{center}
	\texttt{sshpass -p zyboX ssh -o StrictHostKeyChecking=no zyboX@zyboX}
\end{center}
Donde \texttt{X} es el identificador de la placa a la que nos queremos conectar.

Luego, nos situamos en el directorio donde se encuentra el archivo de la primera placa que queramos enviar a la segunda, y escribimos el siguiente comando:
\begin{center}
	\texttt{sshpass -p zyboX scp -o StrictHostKeyChecking=no archivoLocal zyboX@zyboX:/home/zyboX/ficheros/recibir}
\end{center}
Siendo:
\begin{itemize}
	\item \textbf{zyboX:} La tarjeta Zybo a la que queremos enviar el archivo. Por ejemplo: \texttt{zybo1}.
	\item \textbf{archivoLocal:} Nombre del archivo local que queremos enviar.
	\item \textbf{Directorio \texttt{/ficheros/recibir}:} Directorio donde se recibirán los archivos en el proyecto de anonimización\footnote{Si quieremos cambiar el directorio, solo tenemos que cambiarlo al igual que en la nota anterior.}.
\end{itemize}

\subsection{Automático}
Usaremos el script \texttt{Enviando.sh} situado en el directorio \texttt{/ficheros}.

Este script usará el mismo comando que el scp de forma manual, pero estará parametrizado para que lo envíe a la siguiente tarjeta según el nombre de todos los dispositivos de forma correlativa. En caso de que se rompa la correlación, lo enviará al ordenador central (esta comprobación la haremos haciendo un ping a la siguiente tarjtea).

\textcolor{red}{Poner el Enviando.sh}

\end{document}