%%\documentclass[a4paper,12pt,oneside]{llncs}
\documentclass[12pt,letterpaper]{article}
\usepackage[right=2cm,left=3cm,top=2cm,bottom=2cm,headsep=0cm]{geometry}

%%%%%%%%%%%%%%%%%%%%%%%%%%%%%%%%%%%%%%%%%%%%%%%%%%%%%%%%%%%
%% Juego de caracteres usado en el archivo fuente: UTF-8
\usepackage{ucs}
\usepackage[utf8x]{inputenc}

%%%%%%%%%%%%%%%%%%%%%%%%%%%%%%%%%%%%%%%%%%%%%%%%%%%%%%%%%%%
%% Juego de caracteres usado en la salida dvi
%% Otra posibilidad: \usepackage{t1enc}
\usepackage[T1]{fontenc}

%%%%%%%%%%%%%%%%%%%%%%%%%%%%%%%%%%%%%%%%%%%%%%%%%%%%%%%%%%%
%% Ajusta maergenes para a4
%\usepackage{a4wide}

%%%%%%%%%%%%%%%%%%%%%%%%%%%%%%%%%%%%%%%%%%%%%%%%%%%%%%%%%%%
%% Uso fuente postscript times, para que los ps y pdf queden y pequeños...
\usepackage{times}

%%%%%%%%%%%%%%%%%%%%%%%%%%%%%%%%%%%%%%%%%%%%%%%%%%%%%%%%%%%
%% Posibilidad de hipertexto (especialmente en pdf)
%\usepackage{hyperref}
\usepackage[bookmarks = true, colorlinks=true, linkcolor = black, citecolor = black, menucolor = black, urlcolor = black]{hyperref}

%%%%%%%%%%%%%%%%%%%%%%%%%%%%%%%%%%%%%%%%%%%%%%%%%%%%%%%%%%%
%% Graficos 
\usepackage{graphics,graphicx}

%%%%%%%%%%%%%%%%%%%%%%%%%%%%%%%%%%%%%%%%%%%%%%%%%%%%%%%%%%%
%% Ciertos caracteres "raros"...
\usepackage{latexsym}

%%%%%%%%%%%%%%%%%%%%%%%%%%%%%%%%%%%%%%%%%%%%%%%%%%%%%%%%%%%
%% Matematicas aun más fuertes (american math dociety)
\usepackage{amsmath}

%%%%%%%%%%%%%%%%%%%%%%%%%%%%%%%%%%%%%%%%%%%%%%%%%%%%%%%%%%%
\usepackage{multirow} % para las tablas
\usepackage[spanish,es-tabla]{babel}

%%%%%%%%%%%%%%%%%%%%%%%%%%%%%%%%%%%%%%%%%%%%%%%%%%%%%%%%%%%
%% Fuentes matematicas lo mas compatibles posibles con postscript (times)
%% (Esto no funciona para todos los simbolos pero reduce mucho el tamaño del
%% pdf si hay muchas matamaticas....
\usepackage{mathptm}

%%% VARIOS:
%\usepackage{slashbox}
\usepackage{verbatim}
\usepackage{array}
\usepackage{listings}
\usepackage{multirow}

%% MARCA DE AGUA
%% Este package de "draft copy" NO funciona con pdflatex
%%\usepackage{draftcopy}
%% Este package de "draft copy" SI funciona con pdflatex
%%%\usepackage{pdfdraftcopy}
%%%%%%%%%%%%%%%%%%%%%%%%%%%%%%%%%%%%%%%%%%%%%%%%%%%%%%%%%%%
%% Indenteacion en español...
\usepackage[spanish]{babel}

\usepackage{listings}
% Para escribir código en C
% \begin{lstlisting}[language=C]
% #include <stdio.h>
% int main(int argc, char* argv[]) {
% puts("Hola mundo!");
% }
% \end{lstlisting}


\title{Tutorial de instalación de Linux en tarjeta SD para Zybo}
\author{Jesús Rodríguez Heras}

\begin{document}
	
	\maketitle
	\begin{abstract} %Poner esto en todas las prácticas de PCTR
		\begin{center}
			En este documento se desarrolla la explicación de cómo instalar Linux (Xilinux) en las tarjetas SD de las placas Zybo Zynq 7010.
		\end{center}
	\end{abstract}
	\thispagestyle{empty}
	\newpage
	
	\tableofcontents
	\newpage
	
	%%\listoftables
	%%\newpage
	
	%%\listoffigures
	%%\newpage
	
	%%%% REAL WORK BEGINS HERE:
	
	%%Configuracion del paquete listings
	\lstset{language=bash, numbers=left, numberstyle=\tiny, numbersep=10pt, firstnumber=1, stepnumber=1, basicstyle=\small\ttfamily, tabsize=1, extendedchars=true, inputencoding=latin1}

\section{Preámbulo}
Para este proyecto, optamos inicialmente por la instalación de un sistema operativo compilado por nosotros mismos, del cual encontramos una guía en GitHub\footnote{\url{https://github.com/SDU-Embedded/linux_zynq/wiki/Installing-Linux-on-the-ZYBO}.} que explica cómo descargar, compilar y montar el sistema operativo en la tarjeta SD desde cero.

Mientras seguíamos dicho tutorial, solo aparecían errores de compilación debido a incompatibilidad de librerías y dereferenciaciones de las mismas. Concretamente, los errores más importantes fueron:
\begin{itemize}
	\item A la hora de compilar el kernel nos vimos obligados a instalar el paquete ``\texttt{gcc-arm-linux\\-gnueabihf}'', que podemos instalarlo con el siguiente comando:
	\begin{center}
		\texttt{sudo apt-get install gcc-arm-linux-gnueabihf}
	\end{center}
	\item A la hora de compilar el fichero \texttt{U-BOOT}, nos devolvía el siguiente error: ``\texttt{dereferencing pointer to incomplete type ‘rsa aka struct rsa\_st}''. Para solucionarlo, debimos instalar las librerías de desarrollo de SSL con el siguiente comando:
	\begin{center}
		\texttt{sudo apt-get install libssl1.0-dev}
	\end{center}
\end{itemize}
Una vez instaladas las librerías necesarias para solventar los errores existentes, llegó el momento de particionar la tarjeta SD tal y como se muestra en la guía de GitHub. Esto también nos dió serios problemas debido a que, al establecer el nombre de ``\texttt{ROOT\_FS}'' en la partición del sistema de ficheros de linux, no nos dejaba importar los ficheros creados tras la compilación así que los tuvimos que importar como super-usuario en el ordenador.

Cuando la tarjeta SD estuvo preparada, la conectamos a la placa Zybo, pusimos el jumper JP5 en modo SD, conectamos la placa al ordenador por USB, iniciamos la placa y abrimos un terminal de PuTTY\footnote{Puerto ttyUSB1 y velocidad 115200.} en el ordenador, pero no mostraba nada por pantalla.

\section{Decisión final}
Después de probar muchas veces con el método mencionado, optamos por instalar en las placas Zybo el sistema operativo Xilinux que vimos en el blog ``diebotrise\footnote{\url{http://diebotreise.blogspot.com/2017/03/tutorial-zybo-linux-i-como-cargar-linux.html}.}''.

Por lo tanto, seguiremos los pasos de este tutorial:

\subsection{Instalación de Xilinux en tarjeta SD}
Para la instalación de Xiliniux, seguiremos los siguientes pasos:
\begin{itemize}
	\item Descargaremos una versión de Xilinux precompilada desde el siguiente \href{https://www.dropbox.com/s/9qgcoyjzoi764f0/2016.02.02.debian-cbaeef7e7f551052f5451957a5dbef43.zip?dl=0}{\textcolor{blue}{enlace}}.
	\item Descomprimimos el archivo descargado.
	\item Descargamos e instalamos el programa Win32 Disk Imager desde el siguiente \href{https://dw.uptodown.com/dwn/w76tVn7onjw1uZFTLSx7oG5vt1Y7gsfE_vPZCAa88I4YqL5Lp6S__CQhpJZGOLPYdjr73a4yGULPiRxv8Z2IsSRQjRPewPceg1Ol2gDzH3IkO3MHOCcuKQCNZwYI9Pvt/8_NYEhYYhGoJ_QXK-PQtvMDn5aHkqiWxwMofLuT2S5SxxDw2zu6f1OMCW0kqLnB0PGf-zrvou_F_nzjB4fn6Nuvp3WZcVrkFzGgNrhVLInDUMHPM0Jfxh76lJU_IATF7/9xrSlUN8npqZOVysN4LIc5iQnXIPmWNSWKNBLv7hcuQxXmyjX8qERxO48SnXIgr83mVVF-bcUsWJGGryhVqZ_LiBsoZmxxEyZE5JXoqVhCI=/}{\textcolor{blue}{enlace}}.
	\item Ejecutamos Win32 Disk Imager y seleccionamos la imagen descomprimida y la unidad de destino.
	\item Seleccionamos cread ``MD5 Hash'' para comprobar que la descarga no está corrupta. El resultado de dicho hash debe ser: ``cbaeef7e7f551052f5451957a5dbef43''.
	\item Pulsamos el botón ``\texttt{Escribir}'' para flashear la unidad seleccionada.
	\item Descargamos el archivo del kernel desde el siguiente \href{https://uc405c43ce82f6c3032b21ba76bf.dl.dropboxusercontent.com/cd/0/get/AehqJhpiTxZ90sAdGxOFqHxr04BAsug_4RaV6MPX5yBDOQBVaIpDw11X6LdVBi7DPsndN7IxeWYRcEmUAt4xnKoIeYJBRtYu4JtZ_M71as8xbw/file?_download_id=621489597300519201006251662619095482850364296264714456845096204891&_notify_domain=www.dropbox.com&dl=1}{\textcolor{blue}{enlace}}.
	\item Abrimos la carpeta contenedora de la tarjeta SD recién creada.
	\item Sobreescribimos los archivos de esa carpeta por los recién descargados del kernel\footnote{Si no se puede sobreescribir por falta de espacio, es preferible eliminarlos y volver a copiar los de la carpeta kernel dentro.}.
	\item Sacamos la tarjeta SD del ordenador y la introducimos en la placa Zybo.
\end{itemize}

\subsection{Inicio de Xilinux desde la tarjeta SD}
Para iniciar la tarjeta con Xilinux debemos seguir los siguientes pasos:
\begin{itemize}
	\item Insertar la tarjeta SD en la placa Zybo.
	\item Cambiar el jumper JP5 a la posición SD para que arranque desde dicha tarjeta SD.
	\item Conectamos el cable USB de la placa al ordenador y arrancamos la placa.
	\item Abrimos un terminal PuTTY en la consola del ordenador\footnote{Puerto ttyUSB1 y velocidad 115200.} y ahora podemos ver como sí tenemos señal y arranca el sistema operativo.
\end{itemize}

\subsection{Creación de usuarios}
Al ser la primera vez que arrancamos el sistema operativo Xilinux, contamos únicamente con el usuario \texttt{root}, cuya contraseña es \texttt{root}. Por lo tanto, tenemos que crear otro usuario, que será con el que iniciemos sesión en las placas usando el comando:
\begin{center}
	\texttt{adduser ZyboX}
	\\\textcolor{red}{Falta ponerlo mejor porque necesito probarlo en las placas de nuevo para detallarlo mejor.}
\end{center}
Donde \texttt{X} es el identificador de la placa con la que estamos trabajando.

\end{document}