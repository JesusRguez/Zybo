\documentclass[aspectratio=169]{beamer}

%% Juego de caracteres usado en el archivo fuente: UTF-8
\usepackage{ucs}
\usepackage[utf8x]{inputenc}
\uselanguage{spanish}
%Para la identación del español
\usepackage[spanish]{babel}
\usepackage{animate}
\setbeamercovered{dynamic}
\useinnertheme{rectangles}

% There are many different themes available for Beamer. A comprehensive
% list with examples is given here:
% http://deic.uab.es/~iblanes/beamer_gallery/index_by_theme.html
% You can uncomment the themes below if you would like to use a different
% one:
%\usetheme{AnnArbor}
%\usetheme{Antibes}
%\usetheme{Bergen}
%\usetheme{Berkeley}
%\usetheme{Berlin}
%\usetheme{Boadilla}
%\usetheme{boxes}
%\usetheme{CambridgeUS}
%\usetheme{Copenhagen}
%\usetheme{Darmstadt}
%\usetheme{default}
\usetheme{Frankfurt}
%\usetheme{Goettingen}
%\usetheme{Hannover}
%\usetheme{Ilmenau}
%\usetheme{JuanLesPins}
%\usetheme{Luebeck}
%\usetheme{Madrid}
%\usetheme{Malmoe}
%\usetheme{Marburg}
%\usetheme{Montpellier}
%\usetheme{PaloAlto}
%\usetheme{Pittsburgh}
%\usetheme{Rochester}
%\usetheme{Singapore}
%\usetheme{Szeged}
%\usetheme{Warsaw}

%Para la identación del español
\usepackage[spanish]{babel}

\title{Infraestructura de red de nodos cifradores/descifradores AES basada en ApSoC}

% A subtitle is optional and this may be deleted
%\subtitle{Optional Subtitle}

\author{Jesús Rodríguez Heras}
% - Give the names in the same order as the appear in the paper.
% - Use the \inst{?} command only if the authors have different
%   affiliation.

%\institute[Escuela Superior de Ingeniería] % (optional, but mostly needed)
%{
%  \inst{1}%
%  Department of Computer Science\\
%  University of Somewhere
%  \and
%  \inst{2}%
%  Department of Theoretical Philosophy\\
%  University of Elsewhere}
% - Use the \inst command only if there are several affiliations.
% - Keep it simple, no one is interested in your street address.

%\date{25 de abril de 2019}
% - Either use conference name or its abbreviation.
% - Not really informative to the audience, more for people (including
%   yourself) who are reading the slides online

%\subject{Theoretical Computer Science}
% This is only inserted into the PDF information catalog. Can be left
% out. 

% If you have a file called "university-logo-filename.xxx", where xxx
% is a graphic format that can be processed by latex or pdflatex,
% resp., then you can add a logo as follows:

% pgfdeclareimage[height=0.5cm]{university-logo}{university-logo-filename}
% \logo{\pgfuseimage{university-logo}}

% Delete this, if you do not want the table of contents to pop up at
% the beginning of each subsection:
\AtBeginSection[]
{
  \begin{frame}<beamer>{Índice}
    \tableofcontents[currentsection]
  \end{frame}
}
%\AtBeginSubsection[]
%{
%	\begin{frame}<beamer>{Índice}
%	\tableofcontents[currentsection,currentsubsection]
%\end{frame}
%}

% Let's get started
\begin{document}

\begin{frame}
  \titlepage
%  \begin{center}
%  Luis Gutiérrez Flores\\
%Nicolás Ruiz Requejo\\
%Jesús Rodríguez Heras\\
%Arantzazu Otal Alberro\\
%Alejandro Segovia Gallardo\\
%Alejandro José Caraballo García\\
%Gabriel Fernando Sánchez Reina	
%  \end{center}
  
\end{frame}

\begin{frame}{Índice}
%\small
\tableofcontents
\end{frame}
%\normalsize

\section{Introducción}
\subsection{Objetivos}
\begin{frame}{Objetivos}
\textcolor{red}{Aquí hablaríamos de lo que son los objetivos al igual que en la memoria pero mucho más esquemático para que me permita hablar más que lo que es la lectura de la diapositiva por parte del tribunal}
\end{frame}
\subsection{Descripción}
\begin{frame}{Descripción}
\textcolor{red}{Hablamos de la descripción como en la memoria}
\end{frame}
\subsection{Alcance}
\begin{frame}{Alcance}
\textcolor{red}{Hablamos de lo que consiste el alcance, como en la memoria.\\Realmente lo he estructurado todo como en la memoria con la idea de tener una guía que seguir y que todo vaya, más o menos, cogido de la mano.}
\end{frame}

\section{Metodología}
\subsection{Marco teórico}
\begin{frame}{Marco teórico}

\end{frame}
\subsection{Tecnologías a utilizar}
\begin{frame}{Tecnologías a utilizar}

\end{frame}
\subsection{Análisis del sistema}
\begin{frame}{Análisis del sistema}

\end{frame}
\subsection{Diseño y desarrollo}
\begin{frame}{Diseño y desarrollo}

\end{frame}
\subsection{Pruebas del sistema}
\begin{frame}{Pruebas del sistema}

\end{frame}

\section{Conclusiones y trabajo futuro}
\subsection{Conclusiones}
\begin{frame}{Conclusiones}

\end{frame}
\subsection{Trabajo futuro}
\begin{frame}{Trabajo futuro}

\end{frame}

\end{document}


